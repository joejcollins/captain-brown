\documentclass{shortbook}

\title{Adolf Hitler}
\subtitle{His Part in My Childhood}
\author{Michael S Collins}
\date{2026}


\begin{document}

\ShortBookTitlePage


\section*{Preface}

Mike Collins, born in 1930 on the Isle of Wight,
grew up during the Second World War, experiencing air raids, flying bombs,
and the presence of soldiers and prisoners of war.
His father, who lost a right arm and half a lung in South Africa,
was a difficult but capable man, while his mother was reserved.
Mike left school at 15 without qualifications and
often only obtained work by being the sole applicant,
yet he managed to build a successful career in engineering and public works.
He met Molly in Ely and they were immediately compatible,
having a long-lasting marriage and three sons.
During his national service,
he encountered people from a wide range of backgrounds,
which broadened his perspective and proved useful in his later professional life.
He maintained a deep connection to the Isle of Wight,
enjoyed sailing throughout his youth,
and reflected on his life with a pragmatic satisfaction rather than sentimentality.

\clearpage

\section{Church Stretton}

I came to Church Stretton because two of my sons, by chance, happened to have ended up here.
Nick came first because he got a teaching job here, and Joe came, you might say, in his slipstream.
That left us in Norwich, and although it was very pleasant, we realised that it was going to be a disadvantage at one stage.
I think if they hadn't come here, we'd probably have stayed there.

\section{Molly}

I’ve been on my own five years now. Molly was a primary school teacher in Ely and she really enjoyed it.
I would have been an awful teacher; I don’t have the patience.
I met her uncle first. He was a builder's representative, and I got on well with him. Neither of us were particularly sentimental, but it was love at first sight for both of us.
Our first date was on the 31st of March 1956, we were engaged by the 31st of December, and got married six months after that. We were lucky.
I don't think I'd have met anyone who was so compatible. Neither of us were inclined to have a fight, and we never had an argument.

\section{School Days}

I wasn’t a very useful boy; I didn't like school. I was born in 1930, and I started serious school just as the war started. It was bad because most of the male teachers had been called up, and so schooling was disrupted. My school had evacuees from Portsmouth, and looking back on it, I had a pretty poor education.

I was an average sort of student and I didn’t pass my eleven plus, but I was good at history. That was the subject I really enjoyed. I think it had something to do with the school I went to because it was historic. Charles the First lived there for some time at Carisbrook Castle, and he was in prison just up the road. The Isle of Wight hadn't got much going for it, but in terms of historical interest, it was a good place to be. Given the limitations of the war, it was a fairly happy childhood.

My father had four other children, although he didn’t see much of them because he was quite good at falling out with people. Kathleen was the closest to me, but she was seven years older. She was a musician, like my father, and she loved playing the piano. I gave up learning music because the chap who taught me the piano was killed in a German raid on Cowes. It was the 5th of May, 1942, and although the Germans carried out a very professional raid, one of the bombs went astray, and this was how he was killed.

The cinema was very important to me and my friends, and we saw some wonderful films like Brief Encounter. They were all well-made, and they formed the basis of my film-going for subsequent years, especially when I started work at Cowes.

Looking back, I spent a lot of time on my own and I used to read quite a bit. So I think that's where I started a habit of reading, mostly history books. I think if I hadn't gone into engineering, I might have gone into history, but whether it would have led to anything useful, I don't know.

\section{War Breaks Out}

When the war started, I was nine years old. My mother took me and Kathleen on holiday to Tynemouth in August of 1939, and it was apparent then that it was probably our last holiday. We could see sandbags around buildings. Things were getting organized. People had got gas masks and identity cards. Everyone was expecting something to go wrong.

I remember well, sitting on the floor at home on a Sunday, when Chamberlain announced the declaration of war. I was expecting the first bombs to fall within the next five minutes, but of course they didn't. At that time, I was excited because I was nine years old. It was only when the bombing started that I was really frightened, when we were in real danger. That’s when a flying bomb damaged the house badly, blew in the windows, and ruined my sister's piano.

That was 1944, and by that time I was used to war and the sort of events that happened. I realised that it was an unusual time. In fact, I can remember wondering, it must have been about 1943, what did people do during peace time, because I’d grown up during the war.

\section{My Father}

\subsection{A Difficult Start to Life}

My father was the manager of the local insurance company. It wasn't a very demanding job. In fact, playing bowls was his real occupation. My mother was born in West London, and my father came from Lancashire, but they both ended up on the Isle of Wight by accident.

He had rather a strange career. At nineteen, he was wounded at Spion Kop in South Africa during the Boer War. A shell took his right arm off, and for somebody who was hoping to be a military musician, that must have been pretty awful. When he came back to this country, nobody wanted to know and the only job he could get was in insurance. He wasn’t keen on it, but he stayed with it. In many ways, he was quite an exceptional chap because he overcame his disability.

Not only did he lose one arm, he lost one lung, which inhibited his breathing, and he was advised to leave the North of England because the atmosphere was so poor, the weather and the pollution. The Isle of Wight, with its sea air, was much better for his health, and he was offered a job there.

He was an awkward sort of chap, inclined to fall out with employers. He could be extremely difficult. In fact, when I worked for the local authority at Newport on the Isle of Wight, I remember the deputy came to us. She knew what she was doing and I didn't, and I remember her looking up at me and saying, “I thought your father was difficult, but you're serious.” Isn’t that terrible?

When my parents met, my mother was an inspector for the then Ministry of Health, and they had just started the national insurance system. Her job was to check whether employers were paying their contributions, and she was sent to the south of England. She ended up with the vicar in the parish church in Newport, asking him where his insurance payments were. Of course, they weren't paying anything at all, and so she asked who was responsible for their insurance, and he said, “Mr. Collins.” On the face of it, if you met the pair of them, you wouldn't have said they had much in common.

My father had three wives, and my mother was the third. It was all very complicated, and it was never talked about very much, so nobody was quite sure what was going on. There were two sons from his first marriage, and a son and a daughter from his second, and then me. I was the last one.

I always regret his disability because it meant that carpentry work and things like that I wasn't able to partake in. I can't say he wasn't unkind to me, but I was aware that I was missing out. I was the last child, and he probably thought, “Oh, I've already brought up four children.”

His first two wives died in childbirth, so if anybody had a hard life as a single parent, he did. He must have been frightened that my mother would also die giving birth to me, but parents in those days didn't talk much, so I'll never know. He had a rough time generally. On the other hand, losing an arm in South Africa was a blessing in disguise, because his battalion, the second battalion of the King's Own Regiment, got wiped out in August 1914.

He could also be very charming, and working in the insurance business in those days, you had to charm people. He must have had something because he lost an arm and a lung, but he still managed to have three wives and five children. I don’t think I’ve inherited any of his charm; I take after my mother in my manner, she was a bit shy and reserved.

There was very little he couldn't do with one arm, but it was a subject that was never mentioned. You didn't mention it, and if anyone termed him disabled, he would hit the roof. But I think people respected him and were also a bit frightened of him too. By this time, my mother was the welfare officer for the local bus company. Of course, they’d lost all their conductors and had to appoint conductresses instead. She got the job in an unusual way. My parents were walking back from registering for work because women had to work then, and they met the manager of the bus company who Kathleen worked for before she joined the army. He inquired how she was getting on, and he started swearing about having to take on these bloody conductresses and not knowing how he could discipline them. My father just spoke up and said, “Well, Ad will do it for you.” And she did. That's how you got jobs in those days. You didn't fill in application forms and go to interviews.

\subsection{My Father the Politician}

My father became the mayor of Newport, and there was a tradition that at the November market, the mayor always painted the horns of prize bulls gold. He was interested in politics, but as I said, he could be very difficult. You were lucky if you had him on your side, but if not, you were very unlucky. Why anyone voted for him, I don't know. Maybe they were frightened of him.

\subsection{The Home Guard}

Towards the end of the war, he was an adjutant in the Home Guard, with one arm! Anthony Eden had made a broadcast asking for volunteers, and he signed up then, when it was established, and stayed with them until it was disbanded in 1944. He enjoyed it, and I think he was quite a useful character.

My role in the whole thing was, every Saturday morning, to cycle to the Colonel's house with the copy of the weekly orders that had been prepared. The Home Guard were actually a formidable force at the end of the war. There were a lot of them, and they had some pretty sophisticated equipment. If they’d been needed, they would have contributed quite a bit, but they wouldn't have been terribly efficient.

They had good weapons, including one thing they called the Northover Projector. It was like a drainpipe on a tripod, which was supposed to fire a bottle of benzene in a tank. The only problem was that sometimes the bottle went off in the barrel of this thing and exploded.

Sunday was their big day when they had regular officers come and train them in military manoeuvres and battlecraft. I think, in a funny way, they were hoping that nothing happened.

\section{Jack Good}

I didn't know until after the war that our next-door neighbour was one of what we called the auxiliary force, who were really guerrillas, a force to oppose the Germans secretly. This was when France had fallen, and there was the threat of invasion. The auxiliary force were people recruited from local businessmen, farmers, and local residents, who knew the community and could fit in, and could be useful for about twenty-four hours. It was recognised that they wouldn't survive, because of the German attitude. If you ever met my neighbour Jack Good, you would say he was about as far from a hero as you'd find.

He was a manager of the local fruit and vegetable company, but he knew the Isle of Wight and its ways very well, just the sort of chap you wanted on your side. War brings out the best and worst in people, and it was only when the war finished that he mentioned that he'd been in the auxiliary force. None of us neighbours knew.

\section{The Flying Bomb}

Our house was a large terraced house, but it was very badly built; it depended on the houses either side to stay up. When the flying bomb came down, I was in bed. It was half past twelve in the morning, and I heard this rush of air. At first, I just thought it was an ordinary bomb, and then there was this huge explosion, and the house shook. I had a terrier, Sam, who used to sleep on my bed, and when the flying bomb came down, the windows came in and he fell off the bed. I put my head under the sheets, and it was a good job I did because if I hadn't, I would have had my face cut. When I saw Sam, he was completely black because of all the soot that had come down the chimney.

\bookfigure[./images/carisbrooke_road_house.jpg]{123 Carisbrooke Road in 2024}

Luckily, the bomb landed in the only piece of marshy ground in the whole town, but it killed one person, a chap who was on his way home after a night out. I don't know why we got this bomb, because we found out later that it originated from Le Havre, and it was either aimed at Portsmouth or Southampton. By that time in the war, the Germans were only interested in revenge.

My father wasn’t there at the time; he had gone to London for some meeting or other, and my mother was in the next room. Once we sobered up, we went next door and joined them in their Morrison table shelter. They were big tables made of steel with mesh netting. People took out their dining table and put in a Morrison shelter, which could bear the weight of a house. It wasn’t very comfortable for us all under the table, but we didn't know what was going on, whether we were going to get another bomb or not. By this time, it was about three o'clock in the morning. Ever since then, I’ve had a very cautious attitude to the world at night.

\section{Sailing}

When I was a young man, my real interest was sailing, and I used to go with George, who I worked for in the office. He owned the boat, and I was his crew. We sailed a twenty-foot 515 yacht. It was really beautiful and a pleasure to sail. It had a jib, a mainsail, and a spinnaker, and we used to race every Tuesday evening at the Island Sailing Club and Thursday evening at the East Coast Sailing Club. Saturday would be either the London or the Carinthian. We reckoned to sail every possible day and we learnt the hard way. The funny thing was, when we started, each one of us thought the other was an experienced sailor.

At that time, you went to work so that you could go sailing. It took priority over everything. One Tuesday evening, we ripped the jib, which meant that we couldn't race on the Thursday. The jib is quite expensive to mend, and we couldn't afford it, but we gave it to a chap that George had gone to school with, and all he said was, “Come back Wednesday and it'll be finished.” We would never have got that level of service as a stranger.

That was genuinely the happiest period of my life, because you were looking forward to the next race, not to work.

\section{Work Life}

Because I left school at fifteen, I didn’t have the qualifications, but this job appeared in the local surveyor engineer's department, and a friend of my father’s suggested I apply for it. I didn’t know anything about it, but I applied, and as luck would have it, nobody else did. We just made it up as we went along. Many of the jobs I got in my working life I only got, I'm sure, because I was the only person applying. If one other person had come along, I wouldn't have got the job because I'm quite sure I didn't have the right personality.

I actually wanted to go into the aircraft business. At the time, Saunders-Roe in Cowes was a leading firm, and they built some cool airplanes. For some reason, I didn't go in that direction, which was lucky because they were eventually shut down by the Americans.

\section{Designing a War Memorial}

One of the strange things that I got involved in was designing a war memorial. Cowes had a shipyard and aircraft factory, the headquarters of combined operations, and they were all desirable objectives. When Cowes was bombed, about thirty people were killed and buried in a communal grave. When the war was over, they wanted a sort of memorial, and for some reason, they asked me to design it. I had no experience of designing memorials, but I agreed to do it, and by chance, the person who actually built it was somebody who sat next to me at school, Cooper. He was a very good stonemason, and I was very pleased with what he did.

The job at Cowes was easy, in that all I had to do was cycle five miles to get there. When I finished that, I worked in Newport for a bit, and then I went to Ely because a person I'd worked for previously had moved there, and they needed somebody to design a wastewater plant. When I went for the interview, they asked me if I’d designed a wastewater plant before, and I said I hadn’t, but I would like to do it. And I did.

\section{The Flying Fortress with the Smoking Engine}

It was a July evening in 1944 after the invasion. I will always remember it because it was a brilliant summer evening, and I was on my way to meet my friends. This plane had obviously been on a raid over France, and I saw flames coming out of the engine on the port side. It was heading north, and I knew where it was going, to the emergency strip near Southampton, about thirty miles away. There were a number of emergency landing strips around the edge of the country, like Stony Cross, Manston, and Woodbridge, which had extra long and wide runways so they could accept planes in poor shape. As it went over, they fired a red rocket, which indicated that they had wounded on board. It was losing height, and I often wonder what sort of landing it made.

The Flying Fortress was a B-17 bomber, what the Americans produced as an alternative to the Lancaster bomber that we built. It was a very poor aircraft though; it wasn't a fortress at all because it needed a crew of ten to defend it. Looking back on it, it wasn't a great aircraft. It didn't carry a very large load, and when they went wrong, they went spectacularly wrong. The Lancaster bombers were a mixed blessing. They weren't as wonderful as people think. They’re considered to have saved the country, and yes, they did, but they did so after a great deal of alteration.

\section{The Messerschmitt Crash}

I could recognise most of the planes that came overhead. There was one Messerschmitt that came down undamaged, near where I lived, and was put on show in the cattle market. You could pay a bit to sit in the cockpit, which I did, and one thing I’ll always remember about the Messerschmitt was that it smelt of oil, and the cockpit was very small.

The pilot was captured by some farm workers who were in the middle of harvest, and there was a furious debate as to whether he should be offered coffee or tea. Eventually, the decision was taken that he would be offered tea for the simple reason they hadn't got any coffee. At that stage of the war, in 1940, there wasn't the sort of vicious attitude towards the Germans that came later. They were only doing their job.

It wasn’t until recent years that I realised why it had landed undamaged. They had run out of fuel, which was a danger because they were flying at extreme range. It was going to raid the radar station at Ventnor. If it had succeeded, that would’ve been the end of the radar, and we would have lost for certain.

The Germans did an extremely good job in the Battle of Britain. In military terms they deserved it, because they were a better set-up for it. But to fly from northern France to raid the radar stations on the Isle of Wight was long range, and there were no drop tanks that they had later in the war. So when they started a battle over England, they were just running on fumes.

Anti-aircraft defence systems are more complex now. I was in anti-aircraft as part of my national service, and it was technically very interesting. I nearly decided to stay in the services, but then I realised there wasn't much of a career. They continued messing people about, constantly on the edge of being posted somewhere else. There wasn't the stability of employment, so I came out.

\section{The Canadians}

\subsection{The water tank}

Dieppe was August 1942. It was a useless enterprise, but the Canadian nation was anxious to get into the war and do something. We had the 2nd Canadian Division stationed on the Isle of Wight, waiting to go over to Dieppe. They were well paid compared with British troops, and in the evenings, they used to drink a lot and fight. On one celebrated evening, there was a static water tank in the market square for putting out fires. The local police sergeant, Sergeant Eames, was busy regaling these chaps about his prowess as a swimmer, and he was, in fact, a good swimmer. But the Canadians obviously thought they’d try him out, and they put him in the tank. They were anxious for a fight, you see.

\subsection{Raid on Dieppe}

A lot of Canadians joined the army because, in those days, there was high unemployment in Canada. There is one incident which will always remain with me, and it's very sad. One evening, some tanks went by our house, and each one of them had painted on the turret the name of a Canadian province. They rattled down the road, chewed the road surface, and made an awful mess. I wasn’t to know that I would see these tanks long after the war had ended. I was in Cambridge looking through a book of photographs that a German staff photographer had taken after the Dieppe raid, and one photograph was of the beach with those same tanks on it. All of them had been knocked out, and the crews were all killed. Less than two weeks previously, those boys had been fighting in the pubs and chucking the policeman in the water tank. It was very sad.

Dieppe was a trial raid to see how we could land tanks on a hostile shore and withdraw them. Well, they learned that you might be able to land them, but getting them back when somebody's shooting at you isn't easy, and it didn't work. People don’t talk about the raid on Dieppe now. A Canadian historian described it as a raid which was not an occupation but a raid that satisfied the attitudes and ambitions of various groups of people. The Royal Air Force wanted a raid, Churchill badly wanted something, and the Canadians wanted something, but it was a flawed concept. It should never have taken place at all, but because it satisfied so many ideas, people thought they’d try it.

As I said, the Canadians were wanting a fight anyway. In the summer of 1942, we were on the edge of losing the war, and something had to happen, so if it hadn't been Dieppe, it would have been something else.

\section{Raids lighting up the skies}

Most evenings, the Germans would be raiding either Southampton or Portsmouth or both, and we could see the flashes from the guns and everything, and we weren't sure whether it was Portsmouth or Southampton or even Cowes that was being bombed. It was a bit frightening, but you got used to it. I had an uncle who lived on the edge of London near Ealing, and they had a pretty awful war. Bombs landed nearby. Fortunately, they didn't damage themselves, but it must have been quite frightening.

\section{Dickie Dove}

One of the first raids was on the first Monday of the school holiday in 1940. My friend Dickie Dove and I spent the morning sitting on the roof of an old car in his father's smallholding, watching this raid on Portsmouth. We'd never seen an air raid before, and it looked brilliant. There were these puffs of smoke in the sky and bangs and all sorts of things, and we sat there, but we weren't terribly worried. The interesting thing was my parents weren't worried; they didn't seem to worry where we were. I suppose in a way, the war started in such a leisurely way that you were almost inoculated against fear.

Looking back on it, you didn't realise at the time, but those Messerschmitt pilots were as frightened as us. They were so young, and they didn't want to be there as much as others didn't want to be there.

\section{Legacy of the War on People}

One thing about both wars, the first world war and the second, is that the people who survived them were used up. They had given their all. A person I worked with was a committee clerk, and during the war, he was a pilot of a flying boat. Once, when he was on patrol in the Indian Ocean, the navigator had a nervous breakdown and declared that they were lost. Jim had no choice but to sit down and try to work out where they were. He set the aeroplane flying in a circle so that he could do the calculations because nobody else in the crew could, and he got them out of it. He was a nice chap, and I got on well with him, but I was very conscious of the fact that he was tired, and this was 1948. It wasn't so much battle fatigue; it was just that they'd given their all, and there was nothing left. Similarly, I had somebody working for me who had been a prisoner of war, and he was utterly broken, physically and mentally. He could just about do the job. It was awful.

Conversely, I worked for an officer when I was doing my national service who had spent the war in a Japanese prisoner of war camp, and you couldn't meet a more lively, difficult, aggressive person. I felt sorry for the Japanese because it brought out in him not lethargy and lack of interest, but the spirit of, "I'm going to beat these buggers." In a way, adversity brought out characteristics of people that they didn't know they'd had.

\section{National Service}

I was in the artillery when I did my national service. Sometimes it was good fun. Sometimes you died laughing. Sometimes you were really upset. I didn’t realise it at the time, but the experience contributed a lot to my career later on.

At six foot four, I was always the tallest on parade, which meant that you had to stand out there on the parade ground before everyone else came out, which was not very funny. The day I was commissioned, I was there three hours beforehand, getting wet and cold, simply because I was tall. I was the second lieutenant, but again, I only got the job by accident. I think I brought a novel approach to it because you had to go to a selection board and do various tests, and some of my answers were probably a bit unusual. But by that stage of my career, I wasn't worried whether I was commissioned or not; I was there for the fun.

The chap in the next bed to me for a lot of my time there has just retired as Professor of Comparative Religion at St Andrews University. He’s a nice chap. I would have rated him as a good friend, but he was determined to be miserable. He didn't want to be there, and he didn't want to do anything. And he succeeded. I conversely was determined to get as much out of national service as I could. I think the best thing, looking back on it, was meeting people from different societies, like the chap who gave me really wise advice after I’d burnt myself on an iron. As luck would have it, he was a fireman on a locomotive in Southall, and he knew to put Vaseline on it. I don't think I would ever have met somebody of his background if I hadn’t met him on national service.

\section{Jack Taylor}

I met people from shops, people from different parts of the country, people you couldn't understand because of their accent. I remember Jack Taylor from Cumberland. He wanted to call home, but he hadn’t handled a telephone before, so I said I'd do it for him. In those days, it was an interesting experience because the call was transferred up the line, and the further up the line you got, the more impossible it was to understand the accents.

Mind you, Jack did well. We were in Salisbury one time and we were going on leave. I had to get down to Southampton to get the boat to the Isle of Wight, and he had to get right up to Cumberland, and I told him he was going to spend half his leave getting up there. But as luck would have it, a car came along, (he told me this afterwards) and the chap said, "Jump in, I'll give you a lift." And when Jack told him he was going up to Cumberland, he said, "So am I," and he took him all the way. He got home hours before I did.

Hitching a lift was a difficult business; you could stand there, and nothing happened, no one would stop. I had a very happy time, and I got to know Salisbury, and decided I wanted to work there. If I hadn't moved from Salisbury to Norwich, I'd have probably stayed there the rest of my time.

\section{Molly’s Boyfriend}

One of Molly's first boyfriends was a German Messerschmitt pilot. It was after the war, 1946-1947. At that time, she was living in Ipswich, and the children from Northgate School had the opportunity to go to Germany. It was a state school; the headmistress’s brother had been a prominent scientist in the radar world, and she ran it as a private school. It was hard to get in, but Molly was bright.

She was sent to a family down in Bavaria regularly for a few years, just for the summer. She told me there was no animosity at all at that time. It was quite a journey for her; she had to travel to Bavaria on her own from Ipswich, through Germany and through Cologne. She learnt German at school, but nobody would ever call her fluent. The family in Bavaria sort of adopted her, took her out, and to the cathedral and so on. She was a tourist there. It's the only thing Molly and I disagreed on. She was much more pro-German than I was, I suppose, because they’d looked after her.

\section{Looking Back}

I suppose my father must have been proud of me, but it wasn’t the done thing to say it. Sons and fathers didn't communicate very much in those days. I communicate more with my sons and grandsons than I ever did with my father, and that’s a very good thing. There's a lot more hugging these days too, but it’s not just the hugging; it's the fact that, as a senior, you’re prepared to admit that you don't know, and that you get things wrong in a way that my father's generation couldn't. They would see it as a sign of weakness, whereas it’s actually a sign of strength.

Looking back, I don’t have any regrets at all. I suppose that sounds a bit smug, but I was lucky in that I worked in the places that I wanted to, and I did the job that I wanted to, and I did it my way, whether it was designing a wastewater treatment plant or a memorial or a water supply system.

I left the Isle of Wight seventy years ago, in 1955, but I still think of it as home. I've got an Isle of Wight attitude, a small island attitude.



\end{document}